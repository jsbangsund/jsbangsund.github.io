%% cv.tex
%% Copyright 2017 Zeyi Fan
%
% This work may be distributed and/or modified under the
% conditions of the LaTeX Project Public License, either version 1.3
% of this license or (at your option) any later version.
% The latest version of this license is in
%   http://www.latex-project.org/lppl.txt
% and version 1.3 or later is part of all distributions of LaTeX
% version 2005/12/01 or later.
%
% This work has the LPPL maintenance status `maintained'.
%
% The Current Maintainer of this work is Zeyi Fan.
%
% This work consists of only the file cv.tex

\documentclass{article}
\usepackage[top=.4in, bottom=.0in, left=3in, right=0.4in,marginparwidth=2.5in]{geometry}
\usepackage{tikz}
\usepackage{xcolor}
\usepackage[absolute,overlay]{textpos}
\usepackage{fontspec}
\usepackage{titlesec}
\usepackage{pstricks}
\usepackage{amssymb}
\usepackage{paralist}
\usepackage{multicol}

%\usepackage[pdfauthor={Zeyi Fan},
            %pdftitle={Zeyi Fan's Resume},
            %pdfkeywords={}]{hyperref}
\usepackage{hyperref}

\definecolor{mygray}{gray}{0.95}
\definecolor{lightdark}{gray}{0.55}
\definecolor{dark}{gray}{0.3}
\definecolor{skillbg}{gray}{0.7}

\newcommand{\amount}{5.7in}
\setcounter{section}{-1}

\linespread{1.2}
\pagenumbering{gobble}

\renewcommand{\labelitemi}{$\blacksquare$}
\renewenvironment{itemize}[1]{\begin{compactitem}#1}{\end{compactitem}}

% Helpers

\newcommand\twodigits[1]{%
   \ifnum#1<10 0#1\else #1\fi
}

%\newcommand\hl[1]{{\ralewaysb #1}}
\newcommand\hl[1]{#1}


% Fonts

\newfontfamily\raleway{Raleway}
\newfontfamily\ralewaym{Raleway Medium}
\newfontfamily\ralewaysb{Raleway SemiBold}
\newfontfamily\ralewayb{Raleway Bold}
\newfontfamily\ralewayeb{Raleway ExtraBold}
\newfontfamily\ralewaybb{Raleway Black}

\setmainfont{Raleway}


% Styles
\newcommand{\name}[2]{
    \begin{center}
        \Huge{
            \ralewayeb{#1 #2}
        }
    \end{center}
}

\newcommand{\tagline}[1]{
    \begin{center}
        \large{
            \color{dark}
            \ralewaysb{#1}
        }
        \vspace{.2em}
    \end{center}
}

\renewcommand{\thesection}{\twodigits{\arabic{section}}.}

\titleformat{\section}
[hang]
{\ralewaysb\large\color{dark}}
{}
{.0em}
{}
[{\titlerule[0.8pt]}]

\titlespacing*{\section}{0pt}{4pt}{10pt}

\setlength{\TPHorizModule}{1mm}
\setlength{\TPVertModule}{1mm}
\setlength{\parindent}{0mm}

\newcommand{\contactline}[2]{
    \ralewaysb{#1} & \raleway{#2}
}

\newcommand{\skilltype}[2]{
	\ralewaysb{#1} - \raleway{#2}
}

\newcommand{\skill}[2]{
    \vspace{.2em}
    \ralewayb{#1}

    \psset{xunit=0.197\linewidth, yunit=5pt}
    \begin{pspicture}[showgrid=false](5,1)
        \psline[linecolor=skillbg](0,0.5)(5,0.5)
        \psline[linecolor=skillbg,arrows=|-|](1,0.5)(2,0.5)
        \psline[linecolor=skillbg,arrows=|-|](3,0.5)(4,0.5)
        \psline[linecolor=skillbg,arrows=-|](0,0.5)(5,0.5)

        \psline[linecolor=black,arrows=|-|](0,0.5)(#2,0.5)
    \end{pspicture}
}

\newcommand{\block}[3]{
    {\large\ralewayb #1}

    \ralewaym{\color{lightdark}#2}

    \raleway{#3}
}

\newcommand{\blocktwocol}[3]{
\begin{multicols}{2}
	{\large\ralewayb #1} \\
	{\ralewayb #2}
	\columnbreak
	{\raleway #3}
\end{multicols}
}


\begin{document}

% background

\begin{tikzpicture}[remember picture,overlay]
  \fill[mygray] (current page.south west) rectangle ([xshift=-\amount]current page.north east);
\end{tikzpicture}

% Side Bar

\begin{textblock}{58.5}(6,7.7)
    %\name{John}{Bangsund}

    %\tagline{Optical and Electronic Materials Scientist}
    \vspace{+15pt}
    \section{Contact}
    \renewcommand{\arraystretch}{1.1}
    \begin{tabular}{rl}
    	\contactline{Phone}{(763) 516-8759} \\
    	\contactline{Email}{jsb@umn.edu} \\
    	\contactline{Website}{\href{https://jsbangsund.github.io}{jsbangsund.github.io}} \\
    	\contactline{GitHub}{\href{https://github.com/jsbangsund}{@jsbangsund}} \\
    	\contactline{Address}{2015 24th Ave S} \\
    	\contactline{}{Minneapolis, MN 55406}
    	%\contactline{LinkedIn}{\href{https://www.linkedin.com/in/jsbangsund/}{/in/jsbangsund}}
    \end{tabular}

    \section{About}

    \color{dark}
	\begin{flushleft}
    I am an optical and electronic materials scientist with eight years of laboratory experience in photovoltaics and LEDs. 
    I develop software and hardware which enable new measurement techniques and improved data analysis.  
    
    %Through my work on OLED degradation, I have developed
    %I have led three collaborative projects on near-infrared absorbing organic solar cells, stability of organic light-emitting devices (OLEDs), and pattern formation during thin film crystallization of organic semiconductors. I have developed MATLAB and Python programs for image analysis, database management, data acquisition and measurement automation, and optical modeling.
	\end{flushleft}

    \section{Skills}
    	\begin{flushleft}
    	\skilltype{Techniques}{Ellipsometry, atomic force microscopy, X-ray diffraction, electronic device characterization, UV-Vis spectrometry, transient fluorescence, optical microscopy, physical vapor deposition} \\
    	\skilltype{Equipment}{lock-in amplifier, pulse generator, oscilloscope, source measure unit, spectrometer, pulsed and CW lasers, Arduino, cryostat} \\
    	\skilltype{Programming}{Python, MATLAB, LabView, GUI development, National Instruments VISA, Git, MongoDB, Image Processing, Optical modeling}
    	\end{flushleft}

\end{textblock}

% Main
\vspace{-25pt}  % Change this value to adjust the top position of the right col
\name{John}{Bangsund}
\vspace{-14pt}
\tagline{Optical and Electronic Materials Scientist}
\vspace{-12pt}

\section{Education}
%\vspace{-15pt}
\block{University of Minnesota}
{Ph.D. Materials Science | GPA: 3.9 | Expected May 2020}
{
    National Science Foundation Graduate Research Fellowship (2015-2020)
}

\block{Michigan State University}
{B.S. in Materials Science \& Engineering | GPA: 4.0 | May 2015}
{
	Concentration in Polymer Science, B.A. in Humanities, Minor in Spanish \\
	Goldwater Scholarship (2014), Alumni Distinguished Scholarship (2011-2015)
}

\section{Work Experience}

\block{Graduate Research Fellow}
{University of Minnesota | Advised by Prof. Russell Holmes | Nov. 2015 - Present}
{
    \begin{itemize}[-]
		\item Collaborated with DuPont to characterize OLED degradation and understand stability differences between proprietary host materials
		\item Developed techniques based on optical probes to quantify the pathways of OLED degradation, such as changes in quantum yield and excited state kinetics
		\item Designed hardware and software for automation of device lifetime testing
		\item Helped build and maintain database for test data storage and analysis
		%\item Studied periodic pattern formation in organic thin films, for use as self-assembled scattering layers and lasers
		\item Implemented image analysis techniques to study crystallization and pattern formation in organic thin films
		%\item Lab leader in equipment maintenance
		%\item Improved understanding of how crystallization in organic thin films can be engineered to achieve large grains
		%\item Wrote open-source Python programs for automating data acquisition with common lab hardware, calculating OLED light outcoupling efficiency, and measuring crystal growth rate
    \end{itemize}
}

\vspace{.5em}

\block{Ellipsometry Technician}
{UMN Characterization Facility | Minneapolis, MN | Oct. 2016 - Present}
{
	\begin{itemize}[-]
		\item Trained over 40 new users in theory and the principles of operation
		\item Maintained instrument and assisted industry partners in sample analysis
		\item Gave workshop presentations and demos to industry partners %(IPRIME mid-year workshops)
	\end{itemize}
}

\vspace{.5em}

\block{Undergraduate Research Assistant}
{Michigan State University | Advised by Prof. Richard Lunt | Aug. 2012 - July 2015}
{
	%\begin{itemize}[-]
		Initiated and led a project on near-infrared absorbers for organic photovoltaics which resulted in a licensed patent and three journal publications
		%\item Performed anion exchange and column chromatography to synthesize a new series of ionic organic donor materials
		%\item Discovered that anions can shift frontier energy levels, improving open-circuit voltage of small bandgap OPVs
	%\end{itemize}
}

\section{Leadership and Outreach}

\block{Research Mentor}
{University of Minnesota and Michigan State University | 2014 - Present}
{
	Closely mentored and managed one graduate student, four undergraduates, and two high school students, leading to contributions in four publications
	%	\begin{itemize}[-]
	%(Robert Newcomb, Tyler Patrick, Natalia Pajares, Jack Van Sambeek, and Trevor Steiner)
	%\item Trained students in a variety of lab techniques and data analysis, leading to contributions in four publications
	%\item Oversaw two high school students in creating organic lasers for a summer research experience project
	%\item Gave guidance to students applying to Goldwater Scholarship and NSF-GRFP
	%	\end{itemize}
}

\vspace{.5em}

\block{Science Fair Mentor}
{Minnesota Academy of Science, Bdote Middle School | Fall 2018}
{
Mentored four middle school students developing science fair projects%, 4 hr/wk over 12 week program
}

%\block{Shop Volunteer and Local Advisory Board}
%{Minnesota Tool Library | Northeast Minneapolis | May 2017 - Present}
%{
%	%\begin{itemize}[-]
%		Volunteer weekly, helping with shop management and event planning
%	%\end{itemize}
%}

\vspace{.5em}

\block{Student Chapter President}
{Engineers Without Borders, MSU Student Chapter | Sep. 2014 - May 2015}
{
	\begin{itemize}[-]
		\item Managed chapter of 40 members, supervising local and international volunteering projects
		%\item International Project Lead (2013-2014), Fundraising Chair (2012-2014)
		\item Coordinated design, fundraising, report writing, and planning for two international projects
		\item Spanish translator on two trips, helping to construct 14 composting latrines
	\end{itemize}
}
%\ralewaysb Engineers' Choice \raleway at CodeU, Google\hfill Aug 2016
%\newpage
%\begin{tikzpicture}[remember picture,overlay]
%\fill[mygray] (current page.south west) rectangle ([xshift=-\amount]current page.north east);
%\end{tikzpicture}
\newgeometry{top=.4in, bottom=.0in, left=0.4in, right=0.4in}
\section{Journal Publications}
\begin{itemize}
\item[8] \textbf{JS Bangsund}, {\em et al}. Assessing Bimolecular Quenching During Operation of Organic Light-Emitting Devices. In prep.
\item[7] \textbf{JS Bangsund}, {\em et al}. Spontaneous Formation of Aligned, Periodic Patterns During Crystallization of Organic Semiconductor Thin Films. {\em Nature Materials} (2019). DOI:  \href{https://doi.org/10.1038/s41563-019-0379-3}{10.1038/s41563-019-0379-3}
\item[6] \textbf{JS Bangsund}, {\em et al}. Improved Stability in Organic Light-Emitting Devices by Mixing Ambipolar and Wide Energy Gap Hosts. {\em Journal of the Society for Information Display} (2019). DOI:  \href{https://doi.org/10.1002/jsid.761}{10.1002/jsid.761}
\item[5] \textbf{JS Bangsund}, {\em et al}. Origin of Lifetime Enhancement in Mixed Emissive Layer Organic Light-Emitting Devices. {\em ACS Applied Materials and Interfaces} (2018). DOI:  \href{https://doi.org/10.1021/acsami.7b16643}{10.1021/acsami.7b16643}
\item[4]  KW Hershey, \textbf{JS Bangsund}, {\em et al}. Decoupling Degradation in Exciton Formation and Recombination During Lifetime Testing of Organic Light-Emitting Devices. {\em Applied Physics Letters} (2017). DOI:  \href{https://doi.org/10.1063/1.4993618}{10.1063/1.4993618}
\item[3] CJ Traverse, M Young, \textbf{JS Bangsund}, {\em et al}. Anions for Near-Infrared Selective Organic Salt Photovoltaics. {\em Scientific Reports} (2017).  DOI:  \href{https://doi.org/10.1038/s41598-017-16539-3}{10.1038/s41598-017-16539-3}
\item[2] M Young, \textbf{JS Bangsund}, {\em et al}. Organic Heptamethine Salts for Photovoltaics and Detectors with Near-Infrared Photoresponse up to 1600 nm. {\em Advanced Optical Materials} (2016). DOI:  \href{https://doi.org/10.1002/adom.201600102}{10.1002/adom.201600102}
\item[1] \textbf{JS Bangsund}, {\em et al}. Organic Salts as a Route to Energy Level Control in Low Bandgap, High Open-Circuit Voltage Organic and Transparent Solar Cells that Approach the Excitonic Voltage Limit. {\em Adv. En. Mat.} (2016). DOI:  \href{https://doi.org/10.1002/aenm.201501659}{10.1002/aenm.201501659}
\end{itemize}

\section{Conference Presentations}
\begin{itemize}
\item[4] \textbf{JS Bangsund}, {\em et al}. Spontaneous Formation of Aligned, Periodic Patterns During Crystallization of Organic Semiconductor Thin Films. Poster Presentation. Materials Research Society Fall Meeting, Boston, MA. 11/2018.
\item[3] \textbf{JS Bangsund}, {\em et al}. Quantifying Multiple Active Degradation Mechanisms in Mixed Host Organic Light-Emitting Devices. Oral Presentation. Materials Research Society Spring Meeting, Phoenix, AZ. 4/2018.
\item[2] \textbf{JS Bangsund}, {\em et al}. Understanding Improved Lifetime in Mixed Emissive Layer Organic Light-Emitting Devices. Oral Presentation. Optical Society of America Solid-State Lighting Meeting, Boulder, CO. 11/2017.
\item[1] \textbf{JS Bangsund}, {\em et al}. Energy Level Control in Organic Salts for Efficient, Deep Near-Infrared Organic and Transparent Photovoltaics. Oral Presentation. Materials Research Society Spring Meeting, Phoenix, AZ. 3/2016.
\end{itemize}

\section{Patents}
%\begin{itemize}
RR Lunt, \textbf{JS Bangsund}, M Young, and CJ Traverse. {\em ORGANIC SALTS FOR HIGH VOLTAGE ORGANIC AND TRANSPARENT SOLAR CELLS}. PCT/US2016/026169. April, 2016. {\em Licensed by Ubiquitous Energy.}
%\end{itemize}

\section{Interests}
Woodworking, gardening, bouldering, backpacking, biking, saxophone, jazz, and creative writing

\end{document}
